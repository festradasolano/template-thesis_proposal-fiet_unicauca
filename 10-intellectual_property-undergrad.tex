\centering

\textbf{ACTA DE PROPIEDAD INTELECTUAL}

\textbf{UNIVERSIDAD DEL CAUCA}

\textbf{FACULTAD DE INGENIERÍA ELECTRÓNICA Y TELECOMUNICACIONES}

\textbf{ACTA DE ACUERDO SOBRE LA PROPIEDAD INTELECTUAL DEL TRABAJO DE GRADO}

\justify

En atención al acuerdo del Honorable Consejo Superior de la Universidad del Cauca, número 008 del 23 de Febrero de 1999, donde se estipula todo lo concerniente a la producción intelectual en la institución, los abajo firmantes, reunidos el día \underline{\hspace{4ex}} del mes de \underline{\hspace{12ex}} de \underline{\hspace{8ex}} en el salón del Consejo de Facultad, acordamos las siguientes condiciones para el desarrollo y posible usufructo del siguiente proyecto.


Materia del acuerdo: Tesis de Maestría o Doctorado para optar el título de Magíster en / Doctor en \underline{\hspace{0.99\textwidth}}.

Título de la Tesis: \underline{\hspace{0.82\textwidth}} \\ \underline{\hspace{0.99\textwidth}}.

Objetivo de la Tesis: \underline{\hspace{0.8\textwidth}} \\ \underline{\hspace{0.99\textwidth}}.

Duración de la Tesis: \underline{\hspace{0.4\textwidth}}.

Cronograma de actividades: \underline{\hspace{0.4\textwidth}}.

Término de vinculación de cada partícipe en el mismo: \underline{\hspace{0.46\textwidth}} \\ \underline{\hspace{0.99\textwidth}}.

Organismo financiador: \underline{\hspace{16ex}}, naturaleza y cuantía de sus aportes \underline{\hspace{16ex}}, porcentaje de los costos del trabajo \underline{\hspace{16ex}}.

Los participantes de la Tesis, el (los) señor(es) estudiante(s) de Maestría / Doctorado \underline{\hspace{10ex}} \underline{\hspace{20ex}} y \underline{\hspace{20ex}}, identificado(s) con la cédula de ciudadanía número \underline{\hspace{11ex}} y \underline{\hspace{11ex}}, respectivamente, a quien(es) en adelante se le(s) llamara "estudiante(s)", el ingeniero \underline{\hspace{20ex}} en calidad de Director del trabajo de grado, identificado con la cédula de ciudadanía \underline{\hspace{12ex}}, a quien en adelante se le llamará "docente", y la Universidad del Cauca, representada por el ingeniero \underline{\hspace{20ex}}, identificado con la cédula de cuidadanía \underline{\hspace{10ex}}, Decano de la FIET, manifiestan que:

\begin{enumerate}[topsep=0pt, leftmargin=0.6cm, label=\arabic*.-]

    \item La idea original del proyecto es de \underline{\hspace{15ex}} quien la propuso y presentó al Grupo de investigación respectivo \underline{\hspace{20ex}}, que la aceptó como tema para el proyecto de grado en referencia.
    
    \item La idea mencionada fue acogida por el  estudiante como proyecto para obtener el grado de \underline{\hspace{8ex}} \underline{\hspace{22ex}}, quien la desarrollará bajo la dirección del docente.
    
    \item Los derechos intelectuales y morales corresponden al docente y a los estudiantes.
    
    \item Los derechos patrimoniales corresponden al docente, a los estudiantes y a la Universidad del Cauca por partes iguales y continuarán vigentes, aún después de la desvinculación de alguna de las partes de la Universidad.
    
    \item Los participantes se comprometen a cumplir con todas las condiciones de tiempo, recursos, infraestructura, dirección, asesoría, establecidas en el anteproyecto, a estudiar, analizar, documentar y hacer acta de cambios aprobados por el Consejo de Facultad, durante el desarrollo del proyecto, los cuales entran a formar parte de las condiciones generales.
    
    \item El estudiante se compromete a restituir en efectivo y de manera inmediata a la Universidad los aportes recibidos y los pagos hechos por la Institución  a terceros por servicios o equipos, si el comité de Postgrados, previo concepto del Comité de Maestría/Doctorado respectivo declara suspendido el proyecto por incumplimiento del cronograma o de las demás obligaciones contraídas por los estudiantes; y en cualquier caso de suspensión, la obligación de devolver en el estado en que les fueron proporcionados y de manera inmediata, los equipos de laboratorio, de cómputo y demás bienes suministrados por la Universidad para la realización del proyecto.
    
    \item El docente y los estudiantes se comprometen a dar crédito a la Universidad y de hacer mención del Fondo de Fomento de Investigación en caso de existir, en los informes de avance y de resultados, y en registro de éstos, cuando ha habido financiación de la Universidad o del Fondo.
    
    \item Cuando por razones de incumplimiento, legalmente comprobadas, de las condiciones de desarrollo planteadas en el anteproyecto y sus modificaciones, el participante deba ser excluido del proyecto, los derechos aquí establecidos concluyen para él.  Además se tendrán en cuenta los principios establecidos en el reglamento del programa y el acuerdo 035 de 1992 vigente de la Universidad del Cauca en lo concerniente a la cancelación y la pérdida del derecho a continuar estudios.
    
    \item El documento del anteproyecto y las actas de modificaciones si las hubiere, forman parte integral de la presente acta.
    
    \item Los aspectos no contemplados en la presente acta serán definidos en los términos del acuerdo 008 del 23 de febrero de 1999 expedido por el Consejo Superior de la Universidad del Cauca, del cual los participantes del acuerdo aseguran tener pleno conocimiento.

\end{enumerate}

\begin{table}[!h]
    \centering
    \begin{tabular}{m{0.25\textwidth} m{0.5\textwidth}}
        
        & \\
        
        Director & \\
        \cline{2-2}
        & NOMBRE: \\
        
        & \\
        
        Estudiante & \\
        \cline{2-2}
        & NOMBRE: \\
        
        & \\
        
        Decano Facultad & \\
        \cline{2-2}
        & NOMBRE: \\

    \end{tabular}
\end{table}